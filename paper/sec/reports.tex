\section{Reports}
\label{sec:reports}

The bot can send to the controller reports containing \textit{current bot configuration}, \textit{scheduled attacks}, \textit{local system analysis} and \textit{local networks analysis}. Each section is attached to reports, if requested in configuration (see options \texttt{cnfInfo}, \texttt{tgtInfo}, \texttt{sysInfo} and \texttt{netInfo} in \ref{sec:configuration}). Since the first two sections are self-explanatory, we focus here only on local system and networks analysis.\\

The local system analysis includes the following components:
\begin{description}
  \setlength\itemsep{1em}
    \item [browsers] list of browsers installed on system. Browsers are detected by parsing the report of installed applications, returned by system's command of specific OS (e.g. Linux command: "\textit{dpkg -l}").
	\item [hostName] name associated with the host machine.
	\item [kernelVersion] version of the system kernel.
	\item [osArch] name of the system's architecture.
  	\item [osName] name of the operating system.
  	\item [osVersion] version of the operating system.
  	\item [userName] name of the active user on the host.
\end{description}
\;

An example of system analysis is the following:
\begin{description}
	\item 
		\begin{verbatim}
		{
		  "Browsers" : "Google Chrome; Safari;",
		  "HostName" : " MacBook Pro di Michele",
	 	"KernelVersion" : " Darwin 15.5.0",
	 	"OsArch" : "x86_64",
		  "OsName" : "Mac OS X",
		  "OsVersion" : "10.11.5",
		  "UserName" : " Michele Porretta (micheleporretta)"
		}
		\end{verbatim}
\end{description}

The local networks analysis includes the following components:

\begin{description}
  \setlength\itemsep{1em}
  \item [Mac] Mac Address.
  \item [Ip] Ip Address.
  \item [CurrentNetworkInfo] List of network interfaces with their details.
  \item [NetworkStatistics] Statistics about network traffic, classified by protocol.
  \item [TcpConnections] Active Tcp connections.
  \item [UdpConnections] Active Udp connections.
 \end{description}

Note: The use of netInfo command requires a more substantial execution time, because it search the network statistics and the active \textit{TCP} and \textit{UDP} connections.

An example of network analysis is the following:

\begin{description}
	\item 
		\begin{verbatim}
		{
		  "MAC" : "6C-D2-C4-76-FB-24",
		  "ip" : "10.224.219.219",
	 	"currentNetworkInfo" : "[ network info ]",
	 	"networkStatistics" : "[ statistics ]",
	 	"tcpConnections" : "[ active tcp connections ]",
	 	"udpConnections" : "[ active udp connections ]"
		}
	\end{verbatim}
\end{description}