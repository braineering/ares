\section{Configuration}
\label{sec:configuration}

The bot behaviour is ruled by a fully customizable configuration. When no custom configuration is specified, the bot looks for a YAML file named \texttt{config.yaml} in the present working directory. If not there, the bot loads the default one, which is hardly wired in code. For details about how to submit a custom configuration, please refer to \ref{sec:usage}.

We now show the configuration components,the YAML configuration schema to express them and a tool to generate configurations.

\begin{description}
  \setlength\itemsep{1em}

  \item[cnfInfo] if true, reports include the current bot configuration.

  \item[tgtInfo] if true, reports include details about the targets to attack.

  \item[sysInfo] if true, reports include details about the localhost system.

  \item[netInfo] if true, the bot report includes details about the network the localhost is attached to.

  \item[polling] period between consecutive requests to the controller for the next command to execute. The polling period is a random amount of time within a certain time interval. The randomness of period guarantees a greater variance in bots behaviour.

  \item[reconnections] number of times the bot tries to connect to an unreacheable controller.

  \item[reconnectionWait] period between reconnections. The reconnection period is a random amount of time within a certain time interval. The period randomness guarantees a greater variance in bots behaviour.

  \item[proxy] HTTP proxy behind which the bot contacts remote controllers and targets. The default configuration has no proxy.

  \item[sleep] calendar for the sleeping mode. The default configuration has no sleep calendar.

  \item[controllers] list of controllers to contact. The default configuration has an empty list of controller and a code wired fallback controller \texttt{data/samples/controllers/1/bot{init,cmd,log}.json}.

  \item[userAgent] the default User-Agent to use in HTTP interactions both with the controller and targets.
\end{description}

The configuration is given in YAML format. The schema shows some non primitive data types that deserves further attention.

\begin{verbatim}
  cnfInfo: True|False
  tgtInfo: True|False
  sysInfo: True|False
  netInfo: True|False
  polling: time-expression
  reconnections: integer in [0,65535]
  reconnectionWait: time-expression
  proxy: proxy-expression
  sleep: cron-expression
  userAgent: string
  controllers:
    - controller-object
\end{verbatim}

A \texttt{time-expression} represents a temporal interval — e.g. a value in seconds between 3 and 5 seconds. This expression is a string in the form \texttt{min-max:unit}, where \texttt{min} is a positive long, \texttt{max} is a positive long greater than or equal to \texttt{min}, and \texttt{unit} is the string representation of a standard Java TimeUnit\footnote{i.e. NANOSECONDS, MICROSECONDS, MILLISECONDS, SECONDS, MINUTES, HOURS, DAYS.}. If \texttt{min} and \texttt{max} are both equal to a positive long \texttt{amount}, the time interval could be representaed both by the redundant expression \texttt{amount-amount:unit} and by the more compact expression \texttt{amount:unit}.

A \texttt{proxy-expression} represents a HTTP proxy — e.g. the proxy 123.123.123.123 with port 3000. This expression is a string in the form \texttt{address:port}, where \texttt{addres} is an IPv4 address and \texttt{port} is a port number. The expression can also be a string \texttt{none}, meaning that no proxy should be used, and \texttt{null}, meaning that any default proxy should be used.

A \texttt{cron-expression} represents a calendar — e.g. every Wednesday between 10 PM and 11PM. This expression is a standard Unix CRON expression. For details about the standard, please refer to \cite{cron-expression}.

A \texttt{controller-object} represents a bot controller — e.g. a controller with init interface X command interface Y and log interface Z. A controller is expressed in the following form:

\begin{verbatim}
  init: resource expression
  cmd:  resource expression
  log:  resource expression
\end{verbatim}

where a \texttt{resource expression} represents a readable local or remote resource — i.e. a standard file pathname or a remote Web URL.

Here we show a sample configuration for reader's convenience. Other sample configurations can be found in \texttt{data/samples/configurations}.

\begin{verbatim}
  cnfInfo: True
  tgtInfo: False
  sysInfo: True
  netInfo: False
  polling: 10-15:SECONDS
  reconnections: 3
  reconnectionWait: 3-5:SECONDS
  proxy: 123.123.123.123:3000
  sleep: * * * SAT-SUN * ?
  userAgent: BOTAgent
  controllers:
    - init: data/samples/controllers/1/botinit.json
      command: data/samples/controllers/1/botcmd.json
      log: data/samples/controllers/1/botlog.json
    - init: data/samples/controllers/2/botinit.json
      command: data/samples/controllers/2/botcmd.json
      log: data/samples/controllers/2/botlog.json
\end{verbatim}

\subsection{Configuration WUI}
\label{sec:configuration-wui}

\textcolor{green}{\lipsum[1]}

\begin{figure}[tp]
  \centering
  \includegraphics{./fig/acmlarge-mouse}
  \caption{\textcolor{green}{The Web User Interface to build configurations.}}
    \label{fig:configuration-wui}
\end{figure}

\textcolor{green}{\lipsum[1]}
