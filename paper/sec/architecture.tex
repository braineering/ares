\section{Architecture}
\label{sec:architecture}

The botnet implements the architecture showned in \Cref{fig:botnet-architecture}, where the C\&C layer is both implemented by a centralized web server exposing REST interfaces, and simulated by the local file system providing JSON files mimicking web server's responses.

\begin{figure}[tp]
  \centering
  \includegraphics[scale=0.25]{./fig/architecture.eps}
  \caption{The botnet architecture. In the \textit{real scenario}, the bot interacts with a C\&C layer implemented by web server exposing REST APIs — namely a controller. In the \textit{test scenario}, the bot interacts with the local file system, simulating controller's responses with local JSON files. In both scenarios, the bot is configured by a local configuration file, may produce local logs and performs attacks via the Internet.}
    \label{fig:botnet-architecture}
\end{figure}

In a \textit{real scenario}, the bot interacts with the web server through its REST APIs, whilst in a \textit{test scenario} it interacts with the local file system providing JSON files that simulates controller's responses.
In both scenarios, the controller is defined giving three interfaces: the \textit{init interface} is the one from which the bot loads the configuration to join the botnet; the \textit{command interface} from which the bot loads commands to execute; the \textit{log interface} is the one that the bot submits reports to.

Such a model makes our bot is suitable both for \textit{local testing}, where interfaces are local files, and \textit{real bot-controller interaction}, where interfaces are web interfaces.
In the following we often says "the bot receives from the controller" and "the bot sends to the controller", meaning without distinction whetere the bot is interacting with the controller in the real or testing scenario.

The bot is configured providing a YAML configuration file and executes commands given in JSON format. Both can be produced by convenient web-based interfaces presented in \ref{sec:configuration-wui} \ref{sec:commands-wui}, respectively.
