\section{Usage}
\label{sec:usage}

The bot can be run as a Java project inside an IDE, as a Maven project or as a self-contained Jar.
In a real scenario, the program would be delivered via an infected file as a self-contained, minimized and obfuscated Jar. Therefore, here, we will show only this compilation and execution method.
The controller can be built and run as a usual Node.js application packaged with NPM.
The reader may refer to the \texttt{README} file for details on other compilation and execution methods.

In the following, we assume that the execution environment is provided with JDK SE 8.* \cite{jdk}, Maven 3.*, \cite{maven}, Node.js \cite{nodejs}, NPM \cite{npm} amd Bower \cite{bower}. 
We also assume that an Internet connection is available and that both the Jar execution and whichever referenced file do not require any root rights.

\subsection{Bot compilation and execution}
\label{sec:bot-compilation-execution}

The bot is compiled into a self-contained Jar using the Maven packaging procedure, running the command:

\begin{verbatim}
  $[bot-dir]> mvn package -P optimize,skip-tests
\end{verbatim}

The previous command executes the packaging using both the profile \texttt{optimize} and \texttt{skip-tests}.

Since the bot code should always be optimized — i.e. minimized and obfuscated — we recommend providing this stage, activating the \texttt{optimize} profile. The code optimization is realized by running Proguard in the background (downloadable from \cite{proguard}) and may slow down the compilation. If the execution environment doesn't require code optimization, or some compilation speed up is desidered, simply omit the corresponding profile.

Since the project involves more than one hundred JUnit tests, we recommend skipping them to speed up compilation, activating the \texttt{skip-tests} profile. If you considered necessary to perform tests, simply omit the corresponding profile.

The compilation produces a Jar file \texttt{bot-1.0-optimized.jar} in the \texttt{target} folder. Since it is self-contained, it can be run in any system directory. In the following, we refer to this Jar with the shorter name \texttt{bot.jar}.

The bot provides the user with three execution mode: \texttt{default}, \texttt{trace} and \texttt{silent}, which are distinguished by the adopted logging discipline. The \texttt{default} mode prints in console the strictly relevant output and produces log files. The \texttt{trace} mode prints in console a detailed tracing output and produces log files. The \texttt{silent} mode does neither print anything in console nor produce any log file.
To run the program in one of these modes, specify the corresponding options \texttt{--trace} or \texttt{--silent}.

The program behaviour can be customized providing a custom configuration file, specified with the option \texttt{--config}.
If no custom configuration is specified, the bot looks for the configuration file \texttt{config.yaml} in the current working directory\footnote{notice that the \textit{current working directory} is the directory from where JVM is executed, not where the Jar is located.}. If no such file can be found, the bot loads the default configuration\footnote{notiche that the default configuration does not provide the bot with any controller.}.

To run the program with default configuration and default execution mode, run the command:

\begin{verbatim}
  $> java -jar bot.jar
\end{verbatim}

Remember tha,t by default, the bot loads the configuration file \texttt{config.yaml} in the current working directory.

To run the program in \texttt{trace} execution mode, run the command:

\begin{verbatim}
  $> java -jar bot.jar --trace
\end{verbatim}

To run the program in \texttt{silent} execution mode, run the command:

\begin{verbatim}
  $> java -jar bot.jar --silent
\end{verbatim}

To run the program with custom configuration, run the command:

\begin{verbatim}
  $> java -jar bot.jar --config <CONFIG-FILE>
\end{verbatim}

The reader may refer to the program CLI helper for the options usage. We show here the helper output for reader's convenience.

\begin{verbatim}
  BOT version 1.0
  Team: ACM Rome Tor Vergata (http://acm.uniroma2.it)

  A bot showcase.
  Coursework in Computer Security 2015/2016

  usage: BOTNET [-c <CONFIG-FILE>] [-h] [-s] [-t] [-v]
   -c,--config <CONFIG-FILE>   Custom configuration.
   -h,--help                 Show app helper.
   -s,--silent               Activate silent mode.
   -t,--trace                Activate trace mode.
   -v,--version              Show app version.
\end{verbatim}

\subsection{Controller compilation and Execution}
\label{sec:controller-compilation-execution}

The controller is built as a usual NPM package, running the command:

\begin{verbatim}
  $[controller-dir]> npm install
\end{verbatim}

To start the controller, attaching it to the default port 3000, run the command:

\begin{verbatim}
  $[controller-dir]> node controller.js
\end{verbatim}

To start the controller, attaching it to the port \texttt{PORT} and allowing verbosity, run the command:

\begin{verbatim}
  $[controller-dir]> node controller.js --port [PORT] --verbose
\end{verbatim}
