\section{Introduction}
\label{sec:introduction}

This work describes the implementation of a botnet with a centralized command and control (C\&C) layer. Both the bot and the controller are distributed as an open source project at \cite{project-repo}.
In \Cref{sec:botnets} we provide a brief overview about botnets, to better contextualize our work.
In \Cref{sec:architecture} we describe the reference architecture implemented by our botnet.
In \Cref{sec:bot} we show how the bot has been modeled as a finite state automaton, giving the pseudocode of each state.
In \Cref{sec:controller} we show the controller capabilities provided to the attacker to manage the botnet.
In \Cref{sec:configuration} we show how the bot can be configured and a convenient web-based dashboard to generate configuration files.
In \ref{sec:commands} we lists all the available commands, giving their definition and JSON schema, and a web-based dashboard to generate them and submit to the botnet.
In \Cref{sec:reports} we show the host and network analysis the bot can perform.
In \Cref{sec:implementation} we show how the bot and the controller have been implemented, with a focus on adopted technologies and code obfuscation.
In \Cref{sec:usage} we give compilation and usage instructions.
In \Cref{sec:sample-execution} we show a sample execution with the strictly relevant output.
In \Cref{sec:further-improvements} we point out some further improvements both for our bot and controller.
In \Cref{sec:conclusions} we finally give our conclusion on the developed botnet.
